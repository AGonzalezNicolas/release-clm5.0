\section {History File Fields}
\label{sec_history_file}
The following sections discuss both the fields that may currently be
output to CLM3.0 history tapes as well as code modifications that the
user must make to add new user-defined fields to the history tapes.

\subsection {Model history fields}
Tables 9-16 list the fields that currently may be output to a CLM3.0
history tape. By default, these fields are also on the primary history
tape.  The dimensions of each field may include 'time' (days ince the
beginning of the simulation, 'levsoi' (number of soil layers, levsoi =
10) and 'lat' and 'lon' (number of latitude and longitude points,
e.g., lat=64, lon=128 for a T42 simulation), for grid averaged two
dimensional output and 'gridcell', 'landunit', 'column' or 'pft' for
one dimensional output.  Note that the 1d dimension type appearing in
the dimensions entry specifies only the default 1d output type. For
example, 'TSA' will be output by default in column 1d output. However,
that default type may be changed for a given history tape via the
setting of the namelist variable {\bf HIST\_TYPE1D\_PERTAPE}. The
Level column can contain either SL or ML as an entry, denoting a
single-level or multi-level field, respectively. Finally, unless
explicitly specified in the description, all fields are time averaged
over the requested history interval.

\begin{longtable}{|l|p{2.3in}|l|l|l|p{1.0in}|} 
\caption{\label{table_master_field_list} Master Field List - Temperature} \\
\hline
\endhead
\hline
Name & Description & Units & 1d Output & Level & Spatial Validity  \\ 
\hline	\hline	

{\bf TSA} 
& 2 m air temperature 
& K       
& column  
& SL 
& global  \\
\hline	

{\bf TV} 
& vegetation temperature 
& K       
& column  
& SL 
& global \\
\hline	

{\bf TG}
& ground temperature
& K        
& column
& SL 
& global \\
\hline	

{\bf TSOI}
& soil temperature 
& K       
& column  
& ML 
& lakes excluded \\  
\hline	

{\bf TLAKE} 
& lake temperature
& K        
& column
& ML 
& nonlakes excluded \\
\hline	

{\bf TSNOW}
& snow temperature 
& K         
& column
& SL 
& lakes excluded \\ 
\hline	

\end{longtable}

\begin{longtable}{|l|p{2.3in}|l|l|l|p{1.0in}|} 
\caption{\label{master_field_list_srad} Master Field List -Surface Radiation} \\
\hline
\endhead
\hline
Name & Description & Units & 1d Output & Level & Spatial Validity  \\ 
\hline	\hline	

{\bf FSA} 
& absorbed solar radiation   
& watt/m2        
& column
& SL
& global \\ 
\hline

{\bf FSR} 
& reflected solar radiation   
& watt/m2        
& column  
& SL
& global \\ 
\hline

{\bf NDVI} 
& surface normalized difference vegetation index   
& unitless        
& column  
& SL
& global \\ 
\hline

{\bf FIRA} 
& net infrared (longwave) radiation   
& watt/m2      
& column  
& SL
& global \\ 
\hline

{\bf FIRE} 
& emitted infrared (longwave) radiation   
& watt/m2      
& column
& SL
& global \\ \hline

\end{longtable}
      
\begin{longtable}{|l|p{2.3in}|l|l|l|p{1.0in}|} 
\caption{\label{master_field_list_sflux} Master Field List -Surface Energy Fluxes} \\
\hline
\endhead
\hline
Name & Description & Units & 1d Output & Level & Spatial Validity  \\ 
\hline	\hline	

{\bf FCTR} 
& canopy transpiration   
& watt/m2      
& column  
& SL
& global (set to 0 over lakes) \\
\hline

{\bf FCEV} 
& canopy (intercepted) evaporation   
& watt/m2      
& column  
& SL
& global (set to 0 over lakes) \\
\hline

{\bf FGEV} 
& ground evaporation   
& watt/m2      
& column  
& SL
& global  \\
\hline

{\bf FSH} 
& sensible heat   
& watt/m2      
& column  
& SL
& global \\
\hline

{\bf FGR} 
& heat flux into snow/soil (includes snow melt)   
& watt/m2      
& column  
& SL
& global \\
\hline

{\bf FSM} 
& snow melt heat flux   
& watt/m2      
& column  
& SL
& global \\
\hline

{\bf TAUX} 
& zonal surface stress   
& kg/m/s2      
& column  
& SL
& global \\
\hline

{\bf TAUY} 
& meridional surface stress   
& kg/m/s2      
& column  
& SL
& global \\ 
\hline

\end{longtable}
      
\begin{longtable}{|l|p{2.3in}|l|l|l|p{1.0in}|} 
\caption{\label{master_field_list_vegphen} Master Field List - Vegetation Phenology} \\
\hline
\endhead
\hline
Name & Description & Units & 1d Output & Level & Spatial Validity  \\ 
\hline	\hline	

{\bf ELAI} 
& exposed one-sided leaf area index   
& m2/m2      
& column  
& SL
& global \\
\hline 

{\bf ESAI} 
& exposed one-sided stem area index   
& m2/m2      
& column  
& SL
& global \\
\hline 

\end{longtable}
      
\begin{longtable}{|l|p{2.3in}|l|l|l|p{1.0in}|} 
\caption{\label{master_field_list_canopy} Master Field List - Canopy Physiology} \\
\hline
\endhead
\hline
Name & Description & Units & 1d Output & Level & Spatial Validity  \\ 
\hline	\hline	

{\bf RSSUN} 
& sunlit leaf stomatal resistance  (minimum over time interval)    
& s/m      
& column  
& SL
& excludes lakes \\
\hline 

{\bf RSSHA} 
& shaded leaf stomatal resistance (minimum over time interval)  
& s/m      
& column  
& SL
& excludes lakes \\
\hline 

{\bf BTRAN} 
& transpiration beta factor (soil moisture limitation)   
& unitless      
& column  
& SL
& excludes lakes \\
\hline 

{\bf FPSN} 
& photosynthesis   
& unitless      
& column  
& SL
& excludes lakes \\
\hline

\end{longtable}

\begin{longtable}{|l|p{2.3in}|l|l|l|p{1.0in}|} 
\caption{\label{master_field_list_hydro} Master Field List - Hydrology} \\
\hline
\endhead
\hline
Name & Description & Units & 1d Output & Level & Spatial Validity  \\ 
\hline	\hline	

{\bf H2OSOI} 
& volumetric soil water   
& mm3/mm3      
& column
& ML
& excludes lakes \\
\hline

{\bf H2OSNO} 
& snow depth (liquid water equivalent)   
& mm       
& column
& SL
& global \\
\hline

{\bf H2OCAN} 
& intercepted water   
& mm       
& column
& SL
& global (set to 0 over lakes) \\
\hline

{\bf SOILLIQ} 
& soil liquid water 
& kg/m2       
& column
& ML
& excludes lakes \\
\hline

{\bf SOILICE} 
& soil ice 
& kg/m2       
& column
& ML
& excludes lakes \\
\hline

{\bf SNOWLIQ} 
& snow liquid water 
& kg/m2         
& column
& SL
& excludes lakes \\
\hline

{\bf SNOWICE} 
& snow ice 
& kg/m2 
& column
& SL
& excludes lakes \\
\hline

{\bf SNOWDP} 
& snow height 
& m          
& column
& SL
& global \\
\hline

{\bf SNOWAGE} 
& snow age 
& unitless          
& column
& SL
& global (set to 0 over lakes) \\
\hline

{\bf QINFL} 
& infiltration 
& mm/s         
& column
& SL
& global \\
\hline

{\bf QOVER} 
& surface runoff 
& mm/s         
& column
& SL
& global \\
\hline

{\bf QRWGL} 
& surface runoff at glaciers, wetlands, lakes 
& mm/s         
& column
& SL
& global \\
\hline

{\bf QDRAI} 
& sub-surface drainage 
& mm/s         
& column
& SL
& global \\
\hline

{\bf QINTR} 
& interception 
& mm/s         
& column
& SL
& global (set to 0 over lakes) \\
\hline

{\bf QDRIP} 
& throughfall 
& mm/s         
& column
& SL
& global \\
\hline

{\bf QMELT} 
& snow melt 
& mm/s         
& column
& SL
& global \\
\hline

{\bf QSOIL} 
& ground evaporation 
& mm/s         
& column
& SL
& global \\
\hline

{\bf QVEGE} 
& canopy (intercepted) evaporation 
& mm/s         
& column
& SL
& global (set to 0 over lakes) \\
\hline

{\bf QVEGT} 
& canopy transpiration 
& mm/s         
& column
& SL
& global (set to 0 over lakes) \\
\hline

{\bf QCHOCNR} 
& RTM river discharge into ocean 
& m3/s         
& column
& SL
& global (only included if {\bf RTM} defined) \\
\hline

{\bf QCHANR} 
& RTM river flow (maximum subgrid flow) 
& m3/s         
& column
& SL
& global (only included if {\bf RTM} defined) \\
\hline

\end{longtable}

\begin{longtable}{|l|p{2.3in}|l|l|l|p{1.0in}|} 
\caption{\label{master_field_list_check} Master Field List - Water and Energy Balance Checks} \\
\hline
\endhead
\hline
Name & Description & Units & 1d Output & Level & Spatial Validity  \\ 
\hline	\hline	

{\bf ERRSOI} 
& soil/lake energy conservation error 
& watt/m2         
& column
& SL
& global \\
\hline

{\bf ERRSEB} 
& surface energy conservation error 
& watt/m2         
& column
& SL
& global \\
\hline

{\bf ERRSOL} 
& solar radiation conservation error 
& watt/m2         
& column
& SL
& global \\
\hline

{\bf ERRH2O} 
& total water conservation error 
& mm         
& column
& SL
& global \\
\hline

\end{longtable}

\begin{longtable}{|l|p{2.3in}|l|l|l|p{1.0in}|} 
\caption{\label{master_field_list_atm} Master Field List - Atmospheric Forcing} \\
\hline
\endhead
\hline
Name & Description & Units & 1d Output & Level & Spatial Validity  \\ 
\hline	\hline	

{\bf RAIN} 
& rain 
& mm/s         
& gridcell
& SL
& global \\
\hline

{\bf SNOW} 
& snow 
& mm/s         
& gridcell
& SL
& global \\
\hline

{\bf TBOT} 
& atmospheric air temperature 
& K         
& gridcell
& SL
& global \\
\hline

{\bf WIND} 
& atmospheric wind velocity magnitude 
& m/s         
& gridcell
& SL
& global \\
\hline

{\bf THBOT} 
& atmospheric air potential temperature 
& K         
& gridcell
& SL
& global \\
\hline

{\bf QBOT} 
& atmospheric specific humidity 
& kg/kg         
& gridcell
& SL
& global \\
\hline

{\bf ZBOT} 
& atmospheric reference height 
& m         
& gridcell
& SL
& global \\
\hline

{\bf FLDS} 
& incident longwave radiation 
& watt/m2         
& gridcell
& SL
& global \\
\hline

{\bf FSDS} 
& incident solar radiation 
& watt/m2         
& gridcell
& SL
& global \\
\hline

\end{longtable}

Note that for snow related fields (e.g. SNOWLIQ), horizontal averaging
is done only using columns that have snow. In this horizontal
averaging lake subgrid points are excluded. Furthermore, for 
snow related fields, vertical averaging is done by summing only over
valid snow layers.

\subsection{Adding new history fields}
\label{sec_historymod}

Model history output may appear in either xy grid form or native
subgrid form (depending on the value of the namelist variable {\bf
HIST\_DOV2XY}).  Native subgrid output may in turn appear in gridcell,
landunit, column or pft form.  History file output is controlled by
the files {\bf histFldsMod.F90} and {\bf histFileMod.F90}. The module
{\bf histFldsMod.F90} contains the user interface which define the
history fields which are output by the mode whereas {\bf module {\bf
histFileMod.F90} contains routines that create history fields, update
and output the model history buffer during during the course of a
model simulation the course of the model simulation.

If a user wants to add a new field in addition to the ones listed
above to the history output, model code must be modified. First, the
user must determine if the field to be added is a single level or a
multi-level level field.  A single level field will have either pft,
column, landunit or gridcell as its only non-time dimension. A
multi-level field will have pft, column, landunit or gridcell as one
dimension, along with a level field (e.g. soil levels) as a second
non-time dimension. Once this has been determined, the user must only
modify {\bf histFldsMod.F90} in order to add a new history field to
the model output.  A summary of the interface modifications needed to
add new fields will now be provided.  This summary will primarily be
done via concrete examples. 
It is assumed in the following
that the user is completely familiar with Fortran 90 pointer concepts
and syntax.

The interface to add history output fields assumes that all arguments
occur as keyword-value pairs.  The interface also assumes that each
field to be added has a {\bf corresponding entry in clmtype}.  In
other words, each history field must be associated with either a
gridcell, landunit, column or pft level data structure that has
already been defined in {\bf clmtype.F90}.

As an example of the call needed to add a single-level history output
field, the interface call to add canopy evaporation to the history
output is as follows:

\begin{verbatim}
    call add_fld1d (fname='FCEV', units='watt/m^2',  &
         avgflag='A', long_name='canopy evaporation', &
         ptr_pft=clm3%g%l%c%p%pef%eflx_lh_vege, set_lake=0._r8)


   {\it fname}     
         history file field name (required)

   {\it long_name} 
         descriptive name of field (required)

   {\it units}     
         units of field (required)

   {\it avg flag}  
         time averaging flag (required) 
         valid values are "X" (maximum over time), "A" (time average), 
         "M" (minimum over time) or "I" (instantaneous)

   {\it ptr_pft} or {\it ptr_col} or {\it ptr_lunit} or {\it ptr_gridcell} 
         pointer to {\bf clmtype} pft, column, landunit or gridcell arrays (required)

   {\it set_lake}
         value to set lake points to (optional)
         if this value is not set it is assumed that field values over lake points 
         will be provided with a valid value by the model
\end{verbatim}

It is important to note that if the namelist variable {\bf
HIST\_DOV2XY} is set to true, history output will appear in two
dimensional grid form.  Two-dimensional grid output will be obtained
by calculating grid cell averages from the ``type1d'' subgrid
components.
 
The following lists steps that must be taken by the user to add a new
history field to the model. Currently, the upward longwave radiation
above the canopy is not output to the history file. We assume that the
user wants to add this field to the history output at either the
column or pft subgrid level. The following steps assume that a
clmpointer array does not exist for the desired history
variable. Steps 1 and 2 below discuss how such a pointer array is
created. If a clmpointer array already exists for the needed history
variable then only steps 4 and 5 are needed. Step 3 is needed only if
1d output is requested at a subgrid level where spatial averaging is
required and that spatial averaging is not currently in the code.


