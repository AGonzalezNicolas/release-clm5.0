\section{Namelist Parameters}
\label{sec_namelist}

CLM3.0 namelist inputs are presented in sections
\ref{subsec_run} - \ref{subsec_ccsm_namelist} below.  In what follows,
"mode" has values of "offline", "ccsm", "cam" or "all", corresponding
to offline mode, ccsm mode, cam mode, or all the modes.  If a namelist
variable setting is listed as {\bf required}, the value must be set in
the namelist in order for the model to execute successfully.  If a
setting is specified as {\bf required} and the mode is only given as
offline, then that variable must only be specified when running in
offline mode.  For namelist variable settings not listed as {\bf
required}, the code will provide default settings at initialization.
In the following variable descriptions, we refer to examples presented
in section \ref{sec_examples}.

\subsection {Specification of run length, run type, initial run date and run decomposition}
\label{subsec_run}

The following list specifies namelist variables associated with the
definition of run case names, run types (restart, initial or branch),
model time step, and initial run date. 

An initial run starts the model from either initial conditions that
are set internally in the code (referred to as arbitrary initial
conditions) or from an initial conditions dataset (see namelist
variable {\bf FINIDAT}) that enables the model to start from a spun-up
state. 

A restart run is an exact continuation of a previous simulation from
its point of termination.  Output from a restart run should be
bit-for-bit the same as if the previous simulation had not
stopped. Run control variables set in the namelist must be the same as
in the run that is being restarted. 

A branch run is a new case that uses a restart dataset from a previous
simulation to begin the integration. For a branch run, the length of
the history interval and the output history fields do not have to be
the same as in the control simulation. For example, the branching
option can be used to output selected fields more frequently than was
the case in the original run or to add new auxiliary history files to
the model run.

\bigskip
\begin{Ventry}{description}
 \item[{\bf name}] {\bf CASEID}
 \item[description] Case name (short identifier for run) (see ex. 1,2,3).
 \item[type] char*256
 \item[mode] offline, ccsm (obtained from atm in cam mode)	
 \item[default] {\bf required} (must be changed for branch run)
\end{Ventry}
\medskip

\begin{Ventry}{description}
 \item[{\bf name}] {\bf CTITLE}
 \item[description] Case title for use within history files (long identifier).
 \item[type] char*ctitle
 \item[mode] offline, ccsm (obtained from atm in cam mode)     
 \item[default] blank 
\end{Ventry}
\medskip

\begin{Ventry}{description}
 \item[{\bf name}] {\bf NSREST}   
 \item[description]     Run type (0 for initial run, 1 for restart, 3 for branch) 
	(see ex. 1,2,3). 
 \item[type] integer       
 \item[mode] offline, ccsm (obtained from atm in cam mode)    
 \item[default] {\bf required}   
\end{Ventry}
\medskip

\begin{Ventry}{description}
 \item[{\bf name}] {\bf DTIME} 
 \item[description]   Model time step (seconds) (see ex. 1). 
 \item[type] integer          
 \item[mode] offline, must agree with CAM model in ccsm mode, obtained from CAM model directly in cam mode  
 \item[default] {\bf required} (suggested range: 1200-3600 s) 
\end{Ventry}
\medskip

\begin{Ventry}{description}
 \item[{\bf name}]  {\bf STOP\_YMD} and {\bf STOP\_TOD}.
 \item[description]  Stop date and stop time of day
 \item[type]     integer       
 \item[mode]     offline (ending time is obtained from CAM in cam mode and CCSM coupler in ccsm mode)
 \item[default] {\bf required} (if NESTEP  not set)  
\end{Ventry}
\medskip

\begin{Ventry}{description}
 \item[{\bf name}]  {\bf NELAPSE}  
 \item[description]  Elapsed run time in model time steps (positive) or days (negative) (see ex. 2). 
	Value is ignored if {\bf STOP\_YMD, STOP\_TOD} are specified.
 \item[type]     integer       
 \item[mode]     offline (ending time is obtained from CAM in cam mode and CCSM coupler in ccsm mode)
 \item[default] {\bf required} (if NESTEP  not set)  
\end{Ventry}
\medskip

\begin{Ventry}{description}
 \item[{\bf name}]   {\bf NESTEP}   
 \item[description]  Ending run time in model time steps (positive) or days (negative) (see ex. 1). 
	Value is ignored if either {\bf NELAPSE} or {\bf STOP\_YMD, STOP\_TOD} are specified.
 \item[type]  integer       
 \item[mode]     offline (ending time obtained from CAM in cam mode and CCSM coupler in ccsm mode)
 \item[default]  {\bf required} (if NELAPSE not set)  
\end{Ventry}
\medskip

\begin{Ventry}{description}
 \item[{\bf name}]  {\bf NELAPSE}  
 \item[description]  Elapsed run time in model time steps (positive) or days (negative) (see ex. 2). 
 \item[type]     integer       
 \item[mode]     offline (obtained from CAM in cam mode and CCSM coupler in ccsm mode)
 \item[default] {\bf required} (if NESTEP  not set)  
\end{Ventry}
\medskip

\begin{Ventry}{description}
 \item[{\bf name}] {\bf START\_YMD} 
 \item[description]  Start date of run (yyyymmdd format) (see ex. 1). 
 \item[type] integer 
 \item[mode] offline, ccsm  (obtained from CAM model in cam mode) 
 \item[default] {\bf required}      
\end{Ventry}
\medskip

\begin{Ventry}{description}
 \item[{\bf name}] {\bf START\_TOD}  
 \item[description]     Start time of day of run (seconds) (see ex. 1). 
 \item[type]     integer 
 \item[mode]     offline, ccsm  (obtained from CAM in cam mode)  
 \item[default] 0    
\end{Ventry}
\bigskip

\begin{Ventry}{description}
 \item[{\bf name}]  {\bf REF\_YMD} 
 \item[description]  Refernce date for time coordinate (yyyymmdd format).
 \item[type]     integer       
 \item[mode]     offline, ccsm (obtain from CAM in cam mode)
 \item[default] no default value
\end{Ventry}
\medskip

\begin{Ventry}{description}
 \item[{\bf name}]  {\bf REF\_TOD} 
 \item[description]  Refernce reference time of day for time coordinate (secs).
 \item[type]     integer       
 \item[mode]     offline, ccsm (obtain from CAM in cam mode)
 \item[default]  0
\end{Ventry}
\medskip

\begin{Ventry}{description}
 \item[{\bf name}] {\bf clump\_pproc}  
 \item[description] Number of ``clumps'' per process (see section \ref{sec_data_structures}).
 \item[type] integer 
 \item[mode] offline, ccsm  (obtained from CAM in cam mode)  
 \item[default]  1 if OpenMP is disabled \newline
                 number of OpenMP threads (specified in the environment) if OpenMP is enabled.
\end{Ventry}
\bigskip

\begin{Ventry}{description}
 \item[{\bf name}] {\bf brnch\_retain\_casename}  
 \item[description] Flag to retain case name on a branch run. If true, then allow
 	case name to remain the same for branch run. In cam mode, this flag must
	be set for both the cam and clm namelists.
 \item[type]     logical
 \item[mode]     offline, cam, ccsm  
 \item[default]  .FALSE.
\end{Ventry}
\bigskip

\subsection {Specification of model input datasets}
\label{subsec_model_input_data_namelist}
The following list specifies namelist variables associated with model
input datasets.

\bigskip
\begin{Ventry}{description}
 \item[{\bf name}] {\bf FSURDAT}      
 \item[description] Full pathname of surface dataset (see ex. 1,2,3). 
 \item[type] char*256   
 \item[mode] all      
 \item[default] blank 
 \item[notes] raw datasets to generate surface dataset provided 
	are in \$CSMDATA/rawdata surface datasets 
	provided with the distribution are in \$CSMDATA/srfdata 
\end{Ventry}
\medskip

\begin{Ventry}{description}
 \item[{\bf name}]  {\bf FINIDAT} 
 \item[description] Full pathname of initial conditions dataset (see ex. 1,2,3).
 \item[type] char*256   
 \item[mode] all   
 \item[default] blank  
 \item[notes] datasets provided are in \$CSMDATA/inidata 
\end{Ventry}
\medskip
                       
\begin{Ventry}{description}
 \item[{\bf name}] {\bf FPFTCON} 
 \item[description] Full pathname of plant functional type (PFT) 
	physiological constants dataset (see ex. 1,2,3). 
 \item[type] char*256   
 \item[mode] all   
 \item[default] {\bf required} 
 \item[notes] dataset provided is \$CSMDATA/pftdata/pft-physiology 
\end{Ventry}
\medskip

\begin{Ventry}{description}
 \item[{\bf name}] {\bf FRIVINP\_RTM}  
 \item[description] full pathname of RTM input dataset (see ex. 4). 
 \item[type] char*256   
 \item[mode] offline, ccsm   
 \item[default] {\bf required} if cpp token {\bf RTM} is defined in preproc.h  
 \item[notes] dataset provided is \$CSMDATA/rtmdata/rdirc.05 
\end{Ventry}
\medskip

\begin{Ventry}{description}
 \item[{\bf name}] {\bf NREVSN}       
 \item[description] Full pathname of restart file name (only for branch runs) (see ex. 3). 
 \item[type] char*256   
 \item[mode] all   
 \item[default] {\bf required} (only if branch run, NSREST=3) 
\end{Ventry}
\medskip

\begin{Ventry}{description}
 \item[{\bf name}] {\bf MKSRF\_ALL\_PFTS}  
 \item[description] Flag to turn on new surface dataset format. If true,
     new surface dataset format (see section \ref{sec_surface_dataset}) is used.
 \item[type] logical
 \item[mode] all  
 \item[default] .false.
\end{Ventry}
\medskip

\begin{Ventry}{description}
 \item[{\bf name}] {\bf MKSRF\_FVEGTYP}  
 \item[description] Full pathname of raw vegetation type dataset (see ex. 5). 
 \item[type] char*256   
 \item[mode] all  
 \item[default] {\bf required} (if FSURDAT is blank)   
 \item[notes] dataset provided is \$CSMDATA/rawdata/mksrf\_pft.nc 
\end{Ventry}
\medskip

\begin{Ventry}{description}
 \item[{\bf name}] {\bf MKSRF\_FSOITEX}  
 \item[description] full pathname of raw soil texture dataset (see ex. 5). 
 \item[type] char*256   
 \item[mode] all  
 \item[default] {\bf required} (if FSURDAT is blank)   
 \item[notes] dataset provided is \$CSMDATA/rawdata/mksrf\_soitex.10level.nc 
\end{Ventry}
\medskip

\begin{Ventry}{description}
 \item[{\bf name}] {\bf MKSRF\_FSOICOL}  
 \item[description] Full pathname of raw soil color dataset (see ex. 5). 
 \item[type] char*256   
 \item[mode] all  
 \item[default] {\bf required} (if FSURDAT is blank) 
 \item[notes] \$CSMDATA/rawdata/mksrf\_soicol\_clm2.nc 
\end{Ventry}
\medskip

\begin{Ventry}{description}
 \item[{\bf name}] {\bf MKSRF\_FLANWAT}  
 \item[description]  Full pathname of raw inland water dataset (see ex. 5).
 \item[type] char*256  
 \item[mode] all  
 \item[default] {\bf required} (if FSURDAT is blank) 
 \item[notes]  dataset provided is \$CSMDATA/rawdata/mksrf\_lanwat.nc 
\end{Ventry}
\medskip

\begin{Ventry}{description}
 \item[{\bf name}] {\bf MKSRF\_FURBAN}   
 \item[description] full pathname of urban dataset (see ex. 5). 
 \item[type] char*256   
 \item[mode] all 
 \item[default] {\bf required} (if FSURDAT is blank) 
 \item[notes] dataset provided is \$CSMDATA/rawdata/mksrf\_urban.nc 
\end{Ventry}
\medskip

\begin{Ventry}{description}
 \item[{\bf name}] {\bf MKSRF\_FGLACIER} 
 \item[description]  Full pathname of glacier dataset (see ex. 5). 
 \item[type] char*256   
 \item[mode] all  
 \item[default] {\bf required} (if FSURDAT is blank) 
 \item[notes] dataset provided is \$CSMDATA/rawdata/mksrf\_glacier.nc 
\end{Ventry}
\medskip

\begin{Ventry}{description}
 \item[{\bf name}] {\bf MKSRF\_FLAI}     
 \item[description] full pathname of leaf and stem area index, canopy top and bottom height dataset (see ex. 5). 
 \item[type] char*256   
 \item[mode] all  
 \item[default] {\bf required} (if FSURDAT is blank) 
 \item[notes] dataset provided is \$CSMDATA/rawdata/mksrf\_lai.nc 
\end{Ventry}
\medskip

\begin{Ventry}{description}
 \item[{\bf name}] {\bf MKSRF\_OFFLINE\_FNAVYORO} 
 \item[description] 20 min navy orography dataset used to generate land mask (see ex. 5). 
 \item[type] char*256  
 \item[mode] offline   
 \item[default] {\bf required} (if MKSRF\_OFFLINE\_FGRID not set and FSURDAT is blank)  
 \item[notes] dataset provided is \$CSMDATA/rawdata/mksrf\_navyoro\_20min.nc  
\end{Ventry}
\medskip

\begin{Ventry}{description}
 \item[{\bf name}]  {\bf MKSRF\_OFFLINE\_FGRID}    
 \item[description]  Dataset specifying land grid and mask at desired resolution (see ex. 6). 
 \item[type] char*256   
 \item[mode] offline   
 \item[default] blank, {\bf required} (if MKSRF\_OFFLINE\_FNAVYORO not set and fsurdat is blank)  
\end{Ventry}
\medskip

\begin{Ventry}{description}
 \item[{\bf name}] {\bf MKSRF\_OFFLINE\_EDGEN}    
 \item[description] Northern edge of land grid (degrees north) (see ex. 5). 
 \item[type] real  
 \item[mode] offline   
 \item[default] 90. 
\end{Ventry}
\medskip

\begin{Ventry}{description}
 \item[{\bf name}] {\bf MKSRF\_OFFLINE\_EDGEE}    
 \item[description] Eastern edge of land grid (degrees east) (see ex. 5). 
 \item[type] real   
 \item[mode] offline  
 \item[default] 180. 
\end{Ventry}
\medskip

\begin{Ventry}{description}
 \item[{\bf name}] {\bf MKSRF\_OFFLINE\_EDGES}    
 \item[description] Southern edge of land grid (degrees north) (see ex. 5). 
 \item[type] real  
 \item[mode] offline  
 \item[default] -90.  
\end{Ventry}
\medskip

\begin{Ventry}{description}
 \item[{\bf name}] {\bf MKSRF\_OFFLINE\_EDGEW}    
 \item[description] Western edge of grid land (degrees east) (see ex. 5). 
 \item[type] real 
 \item[mode] offline 
 \item[default] -180.  
\end{Ventry}
\medskip

\begin{Ventry}{description}
 \item[{\bf name}] {\bf OFFLINE\_ATMDIR}  
 \item[description] Directory containing atmospheric forcing datasets (see ex. 1,2,3). 
 \item[type] char*256   
 \item[mode] offline   
 \item[default] {\bf required}  
 \item[notes] datasets provided are in directory \$CSMDATA/NCEPDATA  
\end{Ventry}
\bigskip

{\bf FSURDAT} specifies a surface dataset containing time-invariant
land properties such as plant functional types and soil textures and
time-variant properties such as leaf area index. Surface datasets
provided with the distribution are contained in the following
directories:

\begin{center}
\begin{itemize}
\item {\bf \$CSMDATA/srfdata/offine}: offline mode surface datasets
\item {\bf \$CSMDATA/srfdata/cam}: cam mode surface datasets
\item {\bf \$CSMDATA/srfdata/ccsm}: ccsm mode surface datasets
\end{itemize}
\end{center}

\noindent If {\bf FSURDAT} is set to the empty string, a new surface dataset is
generated at run time for the specified model resolution and
land/ocean mask. The creation of a new surface dataset also requires
the specification of the full pathname of the following raw datasets:
{\bf MKSRF\_FVEGTYP}, {\bf MKSRF\_FSOITEX}, {\bf MKSRF\_FSOICOL}, {\bf
MKSRF\_FLANWAT}, {\bf MKSRF\_FURBAN}, {\bf MKSRF\_FGLACIER}, {\bf
MKSRF\_FLAI}.  These datasets are only used for the generation of a
model surface dataset. They are provided with the distribution and are
contained in the input data directory {\bf rawdata}.

In addition to raw datasets, a land/ocean mask is also required for
the creation of a new surface dataset. If the model is run in ccsm or
cam mode, this mask is obtained from either the ccsm flux coupler or
from the cam atmosphere model at startup.  In offline mode, however,
the land/ocean mask can either be calculated from a high resolution
orography dataset by setting the namelist variable {\bf
MKSRF\_OFFLINE\_FNAVYORO} or can be read in from an input dataset via
the setting of the namelist variable {\bf MKSRF\_OFFLINE\_FGRID}.

Subroutines involved in creating a runtime surface dataset reside in the
source code directory {\bf mksrfdata/}. In most cases the creation of a new
surface dataset involves a straightforward interpolation from the raw
dataset resolution to the desired model resolution. For soil texture,
however, averaging would create new soil types. Consequently, the
model determines the dominant soil texture profile per gridcell from
the raw resolution to the desired resolution. 

Once the surface dataset is created, the model code reads the necessary
surface dataset back in so that the same results will be obtained
regardless of whether a run starts from an existing surface dataset or
from one that is created at start up.

The input file, {\bf FINIDAT}, contains values for the time-dependent
variables needed to initialize the model from a spun-up state. If {\bf
FINIDAT} is set to the empty string, the model is internally
initialized to non spun-up values. The setting of the namelist
variable, {\bf HIST\_CRTINIC} (described in the next section), can be
used to generate initial files during a model run. Initial datasets
provided with the distribution are contained in the directory {\bf
\$CSMDATA/inidata\_2.1}. A new tool to provide new spun-up initial datasets
from existing initial dataset at other model resolution is described in
section \ref{subsec_interpinic}.

{\bf FPFTCON} specifies the dataset containing plant functional type
physiological constants. The dataset provided with the distribution is
{\bf \$CSMDATA/pftdata/pft-physiology}.

When the cpp token {\bf RTM} is defined, the RTM river routing scheme
will be invoked in running the model. In this case, {\bf FRIVINP\_RTM}
must be set to a river routing dataset. The dataset provided with the
distribution is {\bf \$CSMDATA/rtmdata/rdirc.05}

{\bf NREVSN} is ignored unless a branch run is specified (i.e., {\bf
NSREST} is set to 3).

In offline mode, time dependent atmospheric forcing data must be read
in periodically. The directory containing these files is given by {\bf
OFFLINE\_ATMDIR}. Forcing datasets provided with the distribution can
be found in {\bf \$CSMDATA/NCEPDATA}. This variable is ignored
in cam and ccsm mode.

\subsection {Specification of history and restart files}
\label{subsec_history_namelist}

The following describes namelist variables associated with history,
restart, and initialization files. In what follows, max\_tapes
denotes the maximum allowable number of different types of history
files (tapes) that the model can produce (currently set to 6) and
max\_flds denotes the maximum number of history fields that may appear
on any given history tape (currently set to 1000).

\bigskip
\begin{Ventry}{description}
 \item[{\bf name}] {\bf HIST\_CRTINIC}     
 \item[description] Frequency with which initial datasets will be 
	generated. \\ Valid values are 'MONTHLY','YEARLY','6-HOURLY','DAILY' or'NONE'. 
 \item[type] char*8     
 \item[mode] offline, ccsm (obtained from CAM in cam mode) 
 \item[default] 'YEARLY'                                                        
\end{Ventry}
\medskip

\begin{Ventry}{description}
 \item[{\bf name}] {\bf HIST\_NHTFRQ(max\_tapes)}     
 \item[description] History tape interval(s) \newline
	(+ for model time steps, - for hours, 0 for monthly ave) (see ex. 4). 
 \item[type] integer array (1:max\_tapes)   
 \item[mode] offline, ccsm (obtained from CAM in cam mode) 
 \item[default] -24                                                             
\end{Ventry}
\medskip

\begin{Ventry}{description}
 \item[{\bf name}] {\bf HIST\_MFILT(max\_tapes)}      
 \item[description] Number of time samples per history tape(s) (see ex. 4). 
 \item[type] integer 
 \item[mode] offline, ccsm (obtained from CAM in cam mode) 
 \item[default] 1                                                               
\end{Ventry}
\medskip

\begin{Ventry}{description}
 \item[{\bf name}] {\bf HIST\_NDENS(max\_tapes)}   
 \item[description] Output tape precision(s). 
	Valid values are 1 (double precision) or 2 (single precision). 
 \item[type] integer    
 \item[mode] all   
 \item[default] 2                                                              
\end{Ventry}
\medskip

\begin{Ventry}{description}
 \item[{\bf name}] {\bf HIST\_DOV2XY(max\_tapes)}   
 \item[description] 
	Per tape spatial averaging flag.  If set to true, produces
	grid-average history fields on output tape. If set to false,
	one-dimensional fields are produced (see ex. 4).
 \item[type] logical    
 \item[mode] all    
 \item[default] .TRUE.  
\end{Ventry}
\medskip

\begin{Ventry}{description}
 \item[{\bf name}] {\bf HIST\_AVGFLAG\_PERTAPE(max\_tapes)}   
 \item[description] Per tape time averaging flag. 
	Valid values are 'A' (average over history period), 'I' (instantaneous), 
	'X' (maximum over history period) or 'M' (minimum over history period). 
 \item[type] char*1
 \item[mode] all    
 \item[default] blank
\end{Ventry}
\medskip

\begin{Ventry}{description}
 \item[{\bf name}] {\bf HIST\_TYPE1D\_PERTAPE(max\_tapes)}   
 \item[description] 
	Per tape one dimensional output type. 
	Only used if one dimensional output is selected for the given tape 
	(via the setting of HIST\_DOV2XY). 
	Valid values are 'GRID', 'LAND', 'COLS', 'PFTS'.
	For example, if one dimensional output is selected for tape 3 and 
	HIST\_TYPE1D\_PERTAPE is set to 'COLS', then all the fields will have 1d 
	column output. If the specified one dimensional output type
	is not defined for a given field, output values will be set to 1.e36 
	for that field.
 \item[type] char*4
 \item[mode] all    
 \item[default] blank
\end{Ventry}
\medskip

\begin{Ventry}{description}
 \item[{\bf name}] {\bf HIST\_EMPTY\_HTAPES}   
 \item[description] 
	If set to true, all the history tapes are empty by default.
	Only variables explicitly listed by the user will be output.
 \item[type] logical
 \item[mode] all    
 \item[default] .FALSE.
\end{Ventry}
\medskip

\begin{Ventry}{description}
 \item[{\bf name}] {\bf HIST\_FINCL1...HIST\_FINCL6(max\_flds)}   
 \item[description] 
	List of fields to include on the respective history tape. See
	tables 9-16 for the list of default fields on the primary
	history tape. Namelist specification can take one of two
	forms.  The user may specify only the name of the field to be
	included on the history tape (in which case the default time
	averaging for that field will be used).  For example,
	HIST\_FINCL2='TV', will add the field TV to the second history
	tape with whatever default time averaging was specified for
	TV. Alternatively, the user may specify the field name,
	followed by a : followed by the time averaging flag desired
	(valid flags are 'I' for instantaneous, 'A' for average, 'M'
	for minimum, and 'X' for maximum). For example,
	HIST\_FINCL2='TV:I'' will add the field TV with instantaneous
	output to the second history tape.  
 \item[type] char*34
 \item[mode] all 
 \item[default] blank
\end{Ventry}
\medskip

\begin{Ventry}{description}
 \item[{\bf name}] {\bf HIST\_FEXCL1..HIST\_FEXCL6(max\_flds)}   
\item[description] 
	List of fields to exclude from the respective history
	tape. The field name must appear in the Master Field List.
	Currently, there is one default history tape, (the primary 
	tape is monthly averaged). See tables 9-16 for more details.
 \item[type] char*32
 \item[mode] all    
 \item[default] blank 
\end{Ventry}
\medskip

\begin{Ventry}{description}
 \item[{\bf name}] {\bf MSS\_IRT}   
 \item[description] 
	Mass store retention period (days) for output datasets (see ex. 4) 
 \item[type] integer    
 \item[mode] offline, ccsm (obtained from CAM in cam mode)  
 \item[default] 0 (i.e., history files will be written to local disk, 
	not the NCAR Mass-Store) 
\end{Ventry}
\medskip

\begin{Ventry}{description}
 \item[{\bf name}] {\bf MSS\_WPASS}   
 \item[description] 
	Mass store write password for output datasets.
 \item[type] char*8     
 \item[mode] all      
 \item[default] blank  
\end{Ventry}
\medskip

\begin{Ventry}{description}
 \item[{\bf name}] {\bf RPNTPATH}   
 \item[description] 
	Full unix pathname of the local restart pointer file. 
 \item[type] char*256              
 \item[mode] all       
 \item[default] lnd.CASEID.rpointer in home directory   
\end{Ventry}
\bigskip

The model writes its own history, restart and initial files. History
files are in netCDF file format and contain model data values written
at user specified frequencies during a model run.  Each field has a
default time averaging flag determining how that field will be
accumulated in time over a given history interval.  The choices are to
record averaged, instantaneous, maximum, or minimum values.  The user
may overwrite this default setting via the namelist variable {\bf
HIST\_FINCLX} where X can equal 1 to 6.  If the user wishes to see a
field written at more than one output frequency (e.g. daily and
hourly), additional history files must be declared containing that
field.  By default, CLM3.0 produces a monthly averaged primary history
file and allows the user to define up to five auxiliary history files.
All files contain grid averaged data unless the namelist variable {\bf
HIST\_DOV2XY} is set to false for a given file.  Primary history files
contain the string 'h0', whereas auxiliary history files contain the
string 'h1', 'h2', 'h3','h4' and 'h5'.  Monthly averaged history files
may be produced for history tape ``t'' by setting {\bf
HIST\_NHTFRQ(t)} to zero.  By default, all time averaged fields on
tape t will be output over the period beginning from the first
timestep of the current month up to and including the last timestep of
that month. Each monthly history file will contain exactly one time
slice of data, regardless of the value of {\bf HIST\_MFILT(t)}.

The model will also periodically create initial netCDF datasets
containing only one dimensional instantaneous values of initial data
fields. The model produces initial datasets either yearly, monthly, or
not at all depending on the setting of the namelist variable {\bf
HIST\_CRTINIC} (the default setting is ``YEARLY'').  These datasets
can be utilized as ``spun-up'' initial conditions.

Restart files are in binary format and can be used only for restart or
branch runs from previous model simulations.  Whenever a restart file
is written, a corresponding local disk restart pointer file is
overwritten. The restart pointer file contains the name of the latest
model restart file. By default, the restart pointer file is placed in
the user's home directory under the name, lnd.CASEID.rpointer. The
user may modify the full pathname of the restart pointer file via
the setting of the namelist variable {\bf RPNTPATH}.

The following table specifies the naming convention used for output
files.  In this table the string yyyy refers to the model year, mm
refers to the model month, dd is the model day and sssss corresponds
to seconds into the model day. Note that for non-monthly history
files, yyyy-mm-dd-sssss corresponds to the first timestamp of data on
the file. {\bf CASEID} is the case identifier set via the namelist
input.

\begin{center}
\begin{description}
\item [CASEID.clm2.r.yyyy-mm-dd-sssss ] restart files
\item [CASEID.clm2.i.yyyy-mm-dd-sssss.nc ] initial files
\item [CASEID.clm2.h[012345].yyyy-mm.nc ] monthly average history files
\item [CASEID.clm2.h[012345].yyyy-mm-dd-sssss.nc ] non-monthly history files 
\end{description}
\end{center}

\noindent History, restart and initialization files can be archived on the NCAR
Mass Storage System (MSS) if the namelist variable {\bf MSS\_IRT} is
set to a value greater than zero.  History, restart and initial files are
archived as follows (where {\bf USERNAME} is the upper-case equivalent
of the user's login name, i.e., the user's root directory on the MSS):

\begin{description} 
 \item [history files] /USERNAME/csm/CASEID/lnd/hist
 \item [restart files] /USERNAME/csm/CASEID/lnd/rest
 \item [initial files] /USERNAME/csm/CASEID/lnd/init
\end{description}
 
\subsection {Specification of input physics variables}
\label{subsec_physics}

\begin{Ventry}{description}
 \item[{\bf name}] {\bf IRAD} 
 \item[description] 
	Frequency of solar radiation calculations 
	(+ for model time steps, - for hours). 
 \item[type] integer   
 \item[mode] offline, must be consistent with 
	CAM2 in ccsm mode obtained from CAM in cam mode   
 \item[default] -1   
\end{Ventry}
\medskip

\begin{Ventry}{description}
 \item[{\bf name}] {\bf CSM\_DOFLXAVE}  
 \item[description] 
	If set to true, flux averaging is performed over 
	the duration set in {\bf IRAD}.   
 \item[type] logical    
 \item[mode] ccsm (must agree with CAM2 setting in atm.setup.csh)  
 \item[default] .TRUE.   
\end{Ventry}
\medskip

\begin{Ventry}{description}
 \item[{\bf name}] {\bf WRTDIA}  
 \item[description] 
	If true, global average 2-m temperature 
	written to standard out (ascii log file of the run) (see ex. 4). 
 \item[type] logical  
 \item[mode] all  
 \item[default] .FALSE. 
\end{Ventry}
\bigskip

\subsection {Specification of RTM River routing}
\label{subsec_rtm}

\begin{Ventry}{description}
 \item[{\bf name}] {\bf RTM\_NSTEPS}  
 \item[description] 
	Number of time steps over which RTM output will be averaged.
 \item[type] integer  
 \item[mode] all  
 \item[default] number of timesteps in 3 hours
\end{Ventry}
\bigskip

\subsection {Specification of cam mode namelist}
\label{subsec_cam_namelist}

When running in cam mode, certain CLM3.0 namelist variables cannot be
set independently.  In particular, any user specification for the
namelist variables, {\bf CASEID}, {\bf CTITLE}, {\bf IRAD}, {\bf
NSREST}, {\bf HIST\_CRTINIC}, {\bf HIST\_NHTFRQ(1)}, {\bf
HIST\_MFILT(1)} and {\bf MSS\_IRT} will be overwritten by values
obtained from CAM2 at startup. All other namelist settings may be set
independently by the user.

The following table specifies the namelist variables that are
overwritten with values obtained from cam and lists the associated CAM2
namelist variable and its default value.

\begin{longtable}{|l|l|l|}
\caption{\label{environment_vars} Namelist Variables overwritten with CAM settings} \\
\hline
CLM Namelist & CAM namelist & CAM default \\ \hline \hline
CASEID & CASEID & required \\ \hline
CTITLE & CTITLE & blank \\ \hline
NSREST & NSREST & 0 \\ \hline
IRAD & IRADSW  & -1 \\ \hline
HIST\_CRTINIC & INITHIST & 'MONTHLY' \\ \hline
HIST\_NHTFRQ(1) & NHTFRQ(1) & 0 \\ \hline
HIST\_MFILT(1) & MFILT(1) & 1 \\ \hline
MSS\_IRT & MSS\_IRT & 365 \\ \hline
\end{longtable}

The minimum set of model input namelist variables that must be given
values by the user at run time depends on if a pre-existing surface
dataset already exists and if RTM river routing is enabled.  At a
minimum, only the namelist variables {\bf FPFTCON} and {\bf FSURDAT}
must be given values if a model surface dataset exists and if the cpp
variable RTM is not defined. If a monthly surface dataset does not
exist, then {\bf MKSRF\_FVEGTYP}, {\bf MKSRF\_FSOITEX}, {\bf
MKSRF\_FSOICOL}, {\bf MKSRF\_FLANWAT}, {\bf MKSRF\_FGLACIER}, {\bf
MKSRF\_FURBAN}, and {\bf MKSRF\_FLAI} must be given values in order to
generate one at runtime.  If RTM is invoked, then {\bf FRIVINP\_RTM}
must also be specified.  Finally, the namelist variable {\bf FINIDAT}
must be set if a pre-existing initial dataset is to be utilized.


\subsection {Specification of ccsm mode namelist}
\label{subsec_ccsm_namelist}

When running in ccsm mode, the user must ensure that settings of the
the namelist variables, {\bf IRAD}, {\bf DTIME} and {\bf
CSM\_DOFLXAVE}, have identical values to the corresponding CAM
namelist variables {\bf IRADSW}, {\bf DTIME} and {\bf FLXAVE}
in the script cam.buildnml\_prestage.csh (see CCSM3.0 User's Guide).

Two input datsets are required in ccsm mode: an RTM input dataset,
rdirc.05, obtained from the namelist variable {\bf FRIVINP\_RTM} and
a dataset containing ecophysiological constants, pft-physiology,
associated with the namelist variable {\bf FPFTCON}.

When running CLM3.0 in ccsm mode, the CAM and CLM grids must be
identical. The flux coupler ensures this by providing the land mask
for this grid at run time as part of the CLM3.0 initialization. This
land mask is dependent on the specific ocean domain utilized.
Consequently, a different surface dataset is required for each atm/ocn
grid combination.  Surface dataset names are automatically generated
by the ccsm clm script given the corresponding atm/ocn
resolutions. These are normally renamed to follow the ccsm surface
dataset naming convention. Currently all supported ccsm-clm surface
datasets will be released with the ccsm distribution.

Additionally, a CLM3.0 spun-up initial dataset may be provided
containing values for the time-dependent variables needed to
initialize the model from a spun-up state by setting the namelist {\bf
FINIDAT}. This file {\bf MUST} have the same atm/ocn resolution and
landmask as the model run for which it will be used.  The variables
appearing in the {\bf \$FINIDAT} file will be internally initialized to non
spun-up values at run time if {\bf FINIDAT} is not set.

Finally, only the namelist variable, {\bf CSM\_DOFLXAVE} is specific
to ccsm mode. If this variable is set to true (the default setting),
flux averaging is performed over the time interval specified by the
namelist variable, {\bf IRAD}.

\subsection {Creation of new spun-up initial dataset}
\label{subsec_interpinic}

Often it may take several simulated years to ``spin up'' the model.
Sometimes it is the case, as when running in CCSM mode, that it may take
numerous decades to spin up the deep soil liquid water. As a result, it
has become necessary to provide users with the tool to create a new
initial dataset at a given resolution using a spun up initial dataset
at either another resolution and/or landmask. In CLM3.0 we are
providing a new tool, {\bf interpinic}, in the directory {\bf
tools/interpinic} that provides users with this functionality.  The
only constraint is that the user must have already created a
``template'' initial dataset at the new resolution before this tool is
used. This can be done by running the model for one day with the
following namelist settings:

\begin{verbatim}
     finidat      = ' '
     hist_crtinic = 'DAILY'
     nelapse      = -24
\end{verbatim}

\noindent
A new initial dataset will be created as a result of this run.  Use of
{\bf interpinic} will then overwrite the non spun-up values of CLM3.0
variables in this initial dataset with spun-up values.
