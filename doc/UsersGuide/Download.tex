\section{Obtaining the Source Code and Datasets}

The source code and datasets required to run the Community Land Model
version 3.0 (CLM3.0) in offline mode (uncoupled from other components
of the Community Climate System Model version 3 (CCSM3.0)) can be
obtained via the web from:

\bigskip
{\bf http://www.cgd.ucar.edu/tss/clm}.
\bigskip

The user should refer to the CAM3.0 User's Guide or the CCSM3.0 General
Documentation for instructions on obtaining code and datasets to run CLM3.0
coupled to other CCSM3.0 components.

It is assumed that the user has access to the utilities {\bf tar},
Free Software Foundation {\bf gunzip} and {\bf gmake (GNU gmake)}.

The CLM3.0 distribution consists of two tar files:

\bigskip
{\bf CLM3.0\_code.tar.gz}
\bigskip

and

\bigskip
{\bf CLM3.0\_inputdata.tar.gz}.
\bigskip

The file CLM3.0\_code.tar.gz contains code, documentation, and
scripts. This file must first be uncompressed with the {\bf gunzip}
utility and then "untarred" as follows:

\bigskip
gunzip -c CLM3.0\_code.tar.gz | tar xvf -
\bigskip

The above command both uncompresses and "untars" the code into the
following subdirectory hierarchy:

\bigskip

\begin{longtable}{|p{1.5in}|p{4.5in}|}
\caption{\label{source_directory} Source Code Directory Structure} \\
\hline
\endhead
\hline
\multicolumn{1}{|c|}{\bf Directory Name} & 
\multicolumn{1}{|c|}{\bf Description}  \\ \hline 
{\bf src/}            &   Directory of FORTRAN and "C" source code \\ \hline
src/biogeophys/       &   Biogeophysics routines (e.g., surface fluxes) \\ \hline
src/biogeochem/       &   Ecosystem and biogechemistry routines \\ \hline
src/csm\_share/       &   Code shared by all the geophysical model components of the Community \\
                      &   Climate System Model (CCSM). Currently contains code for CCSM \\
                      &   message passing orbital calculations, and system utilities. \\ \hline
src/main/             &   Control (driver) routines \\ \hline
src/mksrfdata/        &   Routines for generating surface datasets \\ \hline
src/riverroute/       &   River routing (RTM) routines \\ \hline
src/utils/            &   Independent utility routines \\ \hline
src/utils/esmf/       &   Earth System Modeling Framework utilities \\ \hline
src/utils/timing/     &   General purpose timing library \\ \hline
{\bf bld/}            &   Directory of build, test and run scripts \\ \hline        
bld/offline/          &   Script to build and execute the model on various platforms\\ \hline
{\bf doc/}            &   model documentation \\ \hline
{\bf tools/}          &   Directory of tools for input dataset manipulation \\ \hline
tools/convert\_ascii/ &   Routines for converting user-generated ascii surface dataset files to netCDF format suitable for use by the model 
                          (this tool is used independent of running the model) \\ \hline
tools/interpinic      &   Tool for creating new initial CLM3.0 datasets from existing CLM3.0 datasets at
                          another model resolution and/or landmask. \\ \hline
tools/newcprnc/       &   Tool for comparing model netCDF history files (this tool is used independent of running the model) \\ \hline
\end{longtable}

\bigskip
The file CLM3.0\_inputdata.tar.gz contains surface and offline
atmospheric forcing datasets. This file must first be uncompressed
with the {\bf gunzip} utility and then "untarred" as follows:
\bigskip

	gunzip -c CLM3.0\_inputdata.tar.gz | tar xvf -

\bigskip
The above command results in a directory hierarchy containing {\bf
inputdata/lnd/clm2/} as the root.  This directory hierarchy is
outlined below.
\bigskip

\begin{longtable}{|p{1.5in}|p{4.5in}|}
\caption{\label{inputdata_directory} Input Data Directory Structure} \\
\hline
\endhead
\hline
\multicolumn{1}{|c|}{\bf Directory Name} & 
\multicolumn{1}{|c|}{\bf Synopsis}  \\ \hline 
{\bf NCEPDATA/}        & One year's worth of atmospheric forcing variables
                         in monthly netCDF format suitable for running the model
                         in offline mode (uncoupled from the atmospheric model)  \\ \hline
{\bf inidata\_2.1/}    & Directory hierarchy containing netCDF CLM3.0 initial datasets \\ \hline
{\bf inidata\_2.1/cam} & Initial datasets for initializing CLM3.0 from a spun-up state when
                         running in cam mode (can also be used when running in offline mode) \\ \hline
{\bf inidata\_2.1/ccsm}& Initial datasets for initializing CLM3.0 from a spun-up state when
                         running in ccsm mode (can also be used when running in offline mode) \\ \hline
{\bf inidata\_2.1/offline} & initial datasets for initializing CLM3.0 from a spun-up state when
                             running in offline mode \\ \hline
{\bf pftdata/}         & Plant functional type (PFT) physiological constants
                         dataset (ascii format) \\ \hline
{\bf rawdata/}         & "Raw" (highest provided resolution) datasets (netCDF format) \\ 
                       & (used by CLM3.0 to generate surface datasets at model resolution) \\ \hline 
{\bf rtmdata/}         & River direction map for RTM in ascii format \\ \hline
{\bf srfdata/}         & Directory hierarchy containing netCDF CLM3.0 surface datasets \\ \hline
{\bf srfdata/cam}      & Surface datasets for running CLM3.0 in cam mode \\ 
                       & (can also be used when running in offline mode) \\ \hline
{\bf srfdata/csm}      & Surface datasets for running CLM3.0 in ccsm mode \\ 
                       & (can also be used when running in offline mode) \\ \hline
{\bf srfdata/offline}  & Surface datasets for running CLM3.0 in offline mode \\ \hline
\end{longtable}

\bigskip
